\documentclass[15pt]{article}

\usepackage{lmodern}

\usepackage[document]{ragged2e}
\usepackage{fancyhdr}
\pagestyle{fancy}
\fancyhf{}
\lhead{3012 Homework 07}
\rhead{Daniel Lee}

\rfoot{Page \thepage}
\title{Homework 07}
\author{Daniel Lee }
\date{March 7th,  2022}
\begin{document}
\maketitle
\newpage
\justify
Q1.
9.3: 1, 2, 4b, 5, 6, 7, 10.
Q2.
9.4: 1cdef, 2, 6, 10.
\section*{Question 1}
\subsection*{Problem 1}
$7, 6 + 1, 5 + 2, 5 + 1 + 1, 4 + 3, 4 + 2 + 1, 4 + 1 + 1, 3 + 3 + 1, 3 + 2 + 1 +1, 3 + 1 + 1 + 1 + 1,2 + 2 + 2 + 1, 2 + 2 + 1 + 1 + 1, 2 + 1 +1 +1 +1 +1, 1 + 1 +1 +1+1+1+1$
\subsection*{Problem 2}
a.$f(x)=(\frac{1}{1-x^2})(\frac{1}{1-x^4})(\frac{1}{1-x^4})(\frac{1}{1-x^6})......=\prod_{n=1}^{\infty}\frac{1}{1-x^{2i}}$\\
b.$f(x) = \prod_{n=1}^{\infty}{1+x^{2i}}$\\
c. $f(x) = \prod_{n=1}^{\infty}{1+x^{2i-1}}$
\subsection*{Problem 4b}
$f(t)=\frac{1}{1-t^2} \frac{t^{12}}{1-t^3}\frac{t^{20}}{1-t^5} \frac{t^{35}}{1-t^7}$
\subsection*{Problem 5}
a.$f(t)=1+x^2 +x^4 + x^6 +...= \frac{1}{1 - x^2} $ \\
b.Same as part A.
\subsection*{Problem 6}
a.Since every summand $n \in N$ cannot appear 5 times in the partition.  $f(x)= \prod_{n=1}^{\infty}(1+x^n + x^{2n} + x^{3n} + x^{4n} + x^{5n}) = \prod_{n=1}^{\infty}\frac{1-x^{6i}}{1-x^i}$\\
b.Using the answer obtained in part A, we can find case in which summand cannot exceed 12. $f(x)= \prod_{n=1}^{12}(1+x^n + x^{2n} + x^{3n} + x^{4n} + x^{5n})=\prod_{n=1}^{12}\frac{1-x^{6i}}{1-x^i}$
\subsection*{Problem 7}
$f(x)= \prod_{n=1}^{\infty}\frac{1-x^{3k}}{1-x^k}= \prod_{n\not| \:3}^{\infty}(1-x^l)^{-1}, k,l \in N$\\
g(x) =$\prod_{n\not| \:3}^{\infty}(1+x^l + x^{2l} + x^{3l} + x^{4l}....)= \prod_{n\not| \:3}^{\infty}(1-x^l)^{-1}$
f(x) = g(x). \\Therefore the condition for the problem has been proven true. 
\subsection*{Problem 10}
Let us consider a Ferrers Graph with partition of 2n into n rows. If we were to remove the first column of the graph, the resulting graph would be a partition for n. This is a one to one correspondence thus confirming the number of partition of n is equal to the partition of 2n into n summands.
\newpage
\section*{Question 2}
\subsection*{Problem 1}
c.$e^{-ax}$\\
d.$e^{a^{2x}}$\\
e.$ae^{a^{2x}}$\\
f.$xe^{2x}$\\
\subsection*{Problem 2:}
a.$3,3^2,3^3,3^4,3^5,...$\\
b.$3,24,138,....,6(5^n)- 3(2^n)$\\
c.1,1,3,1,1,1,1,...
d.1,9,14,-10,$2^4,2^5,2^6,..$
e.0!,1!,2!,3!...
f.4,7,25,145,..,(3n!)$2^n + 1$
\subsection*{Problem 6:}
a.i.${(1+x)}^2 (1+x+(\frac{{x^2}}{2!}))^2 ()$\\
ii. $(1+x)(1+x+(\frac{x^2}{2!}))(1+x+(\frac{x^2}{2!})+(\frac{x^2}{3!})+ (\frac{x^2}{4!}))^2$ \\
iii. ${(1+x)}^3(1+x+ \frac{x^2}{2!} )^4$\\
b.$(1+x)\cdot (1+x + \frac{x^2}{2!})\cdot (1+x + \frac{x^2}{2!}+  \frac{x^3}{3!} +  \frac{x^4}{4!}) \cdot ( \frac{x^2}{2!}+  \frac{x^3}{3!} +  \frac{x^4}{4!})$
\subsection*{Problem 10:}
a. $f(x) = (x + \frac{x^3}{3!}+ \frac{x^5}{5!}+..)\cdot (x+ \frac{x^2}{2!} + \frac{x^3}{3!}+...)\cdot {(e^x)}^2 = \frac{1}{2}\cdot {(e^x-1)^2}\cdot {e^{2x}} = \frac{1}{2}(e^{x}-1)\cdot ({e^{3x}-e^{x}}) = \frac{1}{2} (e^{4x}-e^{3x}-e^{2x}+e^{x}) $
Answer is coeff($\frac{x^{20}}{20!}$) = $\frac{1}{2}(4^{20}-3^{20}-2^{20}+1)$ \\
b. $(1+x+\frac{(x^3)}{3!}+ \frac{x^4}{4!}+...)^4 = (e^x -\frac{x^2}{2})^4 = e^{4x}-{4\choose 1} e^{3x}(\frac{x^2}{2}-{4\choose 2}e^{2x}(\frac{x^2}{2})^2)- {4\choose 3}e^{x}(\frac{x^2}{2})^3) + (\frac{x^2}{2})^4$ \\
Answer: $4^{20} - {4\choose 1}\frac{1}{2}3^{18}(20)(19) + {4 \choose 2}\frac{1}{4}(2^{16})(20)(19)(18)(17) - {4 \choose 3}(\frac{1}{8})(20)(19)(18)(17)(16)(15)$
c.$h(x) = (1 + x + (x^3/3!)+(x^4/4!)+....)^4 = (e^x - (x^2/2))^4 = e^{4x} - {4 \choose 1}e^{3x}{(x^3)/6}+ {4 \choose 2}e^{2x}(x^3/6)^2 - (x^3/6)^4\\coeff($\frac{x^{20}}{20!}$) = 4^{20} - {4 \choose 1}(1/6)3^{17}(20)(19)(18) + {4\choose 2 }(1/6)^2(2^{14})(20)(19)(18)(17)(16)()15 - {4\choose 3}(1/6)^3(20!/11!)$\\
d. $(e^x)^3(1+(x^2/2!)) = e^{3x} + e^{3x}(x^2/2!)\\coeff = 3^{20} + (1/2)(1/6)(3^{18})(20)(19) $


\end{document}



